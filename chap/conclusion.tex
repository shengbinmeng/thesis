\chapter{总结和展望}

\section{本文工作总结}

本文围绕视频流媒体这一热点领域展开了较为系统和深入的研究。首先,为了应对异构网络环境中信道带宽的波动,本文研究了基于可伸缩视频编码的自适应流媒体技术,提出了新的可伸缩视频码流截取方案和传输过程中的码率自适应算法。其次,为了在计算资源有限的终端设备上实时解码所收到的码流,本文为最新的国际视频编码标准HEVC设计实现了一个高效的软件解码器,提出了新的优化算法,显著提高了解码速度。本文所取得的主要工作成果包括:
\begin{enumerate}
\item {采用线性误差模型的码流截取方案:}
作为码率适应带宽波动的前提条件,视频流媒体中的数据源需要能够灵活调整。可伸缩视频编码通过将数据划分为基本层和增强层,通过丢弃增强层的数据包来实现即时码率变化。从完整的可伸缩码流中丢弃部分数据得到一个子流的过程称为码流截取。本文以最小化特定截取码率限制下的视频失真为目标,首先提出了一个线性误差模型来估计丢弃任意数据包组合带来的失真变化,然后利用它设计了一个贪心型算法来根据每个数据包的码率和失真影响对其赋优先级,作为截取过程中丢包的顺序。实验表明,本文提出的线性误差模型对失真估计的误差率只有5\%,而采用该模型的码流截取相比于参考软件JSVM中的截取器有高达0.5dB的PSNR提升。
\item {基于PID控制思想的码率自适应算法:}
自适应流媒体的另一个关键问题是传输过程中的码率调整策略,即在可用带宽不断变化的情况下,决定何时调整码率并确定调整到多少。本文基于经典的比例-积分-微分(PID)控制思想,提出了一个综合考虑带宽的历史状况、当前状态和未来趋势的码率自适应算法,既能充分利用带宽,传输较高的视频质量,又能减小带宽波动的影响,保证视频质量的平滑性。该算法在点播和直播的实际测试中都表现出了很好的性能,而且很容易扩展到各种自适应流媒体系统。在达尔文流媒体服务器中实现的点播系统码率自适应算法与内置的算法相比,平均质量等级提高了8.6\%,而质量波动降低了24.8\%。不仅发送视频的质量更高,还取得了更好的平滑性。在直播系统中的实验表明,带宽利用率和播放的连续性都得到了改善。
\item {高度优化的HEVC解码器设计与实现:}
视频流媒体的最后一个阶段是码流在用户终端设备上的解码播放。为此,本文设计并实现了一个高度优化的HEVC软件解码器,将数据级和任务级并行方法相结合,显著提高了各个解码模块的计算效率以及整体解码速度。与业界知名的多媒体框架FFmpeg中的HEVC解码器实现相比,本文优化过的解码器在单线程和多线程情况下分别取得了高达2.8和3.7倍的解码加速比。该解码器分别在主流的桌面和移动处理器上达到了4K ($3840 \times 2160$) 和720p ($1280 \times 720$) 视频的实时解码要求,并已成功应用于业界知名的视频流媒体平台迅雷看看,为进一步改善用户在线观看视频的体验提供了技术支撑。
\end{enumerate}

\section{未来工作展望}

以本文的成果作为基础,后续的研究可以扩展到其他类型的流媒体系统,以及立体视频、虚拟现实等更为复杂的应用场景。下面分三个方面对未来工作进行展望。
\begin{enumerate}
\item {更广泛的数据源优先级划分:}
本文的码流截取工作只是针对可伸缩视频编码的码流,但事实上,流媒体系统中的源端数据多种多样。从目前的趋势来看,视频正朝着立体和全景等更加生动也更为复杂的方向发展,这其中除了传统的图像数据,还有深度数据、空间几何数据等,它们对视频质量的贡献方式与大小都差异很大。为了高效自适应地传输这些数据,弄清楚各个数据包对最终用户观感质量的重要性意义重大。这就是广义上的优先级划分问题。根据不同数据的性质考察其对信息重建的影响,为其建立失真模型并确定传输时的优先级,将是未来研究的内容之一。
\item {PID控制思想的扩展应用:}
PID作为能有效控制和调节各种系统行为的基本思想,具有非常广阔的应用潜力。在本文中,基于PID为达尔文流媒体服务器设计了码率自适应算法;在后续研究中,可以将其应用到DASH等其他流媒体系统中。此外,PID的关键在于反馈,即通过从系统获取信息来反作用于系统的输入,这些信息除了读取系统状态外,还可以从客户端或者观看者收集。尤其是对于虚拟现实应用来说,用户与系统更为紧密,可以成为这一研究方向的用武之地。进一步地,以本文对PID的应用作为启发,在后续工作中可以继续探索如何将非线性控制、最优控制等现代控制论中的原理和方法用于视频和多媒体相关领域。
\item {复杂重建及后处理过程的优化:}
本文所优化的仅仅是从HEVC码流解码出视频图像的的过程,但是客户端的重建和显示所包含的操作还有很多。例如,对于立体视频而言,利用码流中已有的多个视图来合成新视角下的视图就是另一个重要的问题;很多客户端在显示之前,还会对视频图像进行去噪、改善对比度等后处理过程。上述这些操作与本文已经涉及的解码操作类似,计算量大且适合用SIMD技术和并行框架进行加速,而且对于用户的视频播放速度和质量有很大的影响,可以作为下一步优化研究的内容。此外,本文只涉及了通用处理器上的软件优化技术,后续研究还包括用GPU或特定硬件进行加速的方案。
\end{enumerate}

目前虚拟现实的热潮方兴未艾,视频流媒体作为其重要的技术支撑在很长一段时间内都将会有大量新的问题需要解决。服务器上的数据源处理、网络上的传输、客户端的解码重建,这些本文所涉及的系统模块不会有大的改变,而且本文提出的思想方法很容易在新的应用场景或数据形式上适配。现有的这些工作和未来的研究将为人们突破时间和空间的限制更好感受这个世界带来更多可能。