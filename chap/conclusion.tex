\chapter{总结和展望}

\section{本文工作总结}

本文围绕自适应视频流媒体这一热点领域展开了较为系统和深入的研究。首先,本文针对可伸缩视频数据源提出了新的失真模型和码流截取方案,在支持可变码率的同时提供尽可能高的视频质量;其次,本文为基于可伸缩视频编码的视频点播系统设计了新的码率自动调整策略,用控制论的方法来解决如何适应带宽变化的问题;最后,本文详细分析了现在非常流行的视频直播系统的传输过程,结合直播的特点提出了数据上传时的码率自适应算法。总结来说,本文取得的主要工作成果包括以下三个部分。
\begin{enumerate}
\item {采用线性误差模型的可伸缩视频码流截取方案} \\
作为码率适应带宽波动的前提条件,视频流媒体中的数据源需要能够灵活调整。可伸缩视频编码将数据划分为基本层和增强层,通过丢弃增强层的数据包来实现即时码率变化。从完整的可伸缩码流中丢弃部分数据得到一个子流的过程称为码流截取。本文以最小化特定截取码率限制下的视频失真为目标,首先提出了一个线性误差模型来估计丢弃任意数据包组合带来的失真变化,然后利用它设计了一个贪心型算法来根据每个数据包的码率和失真影响对其赋优先级,作为截取过程中丢包的顺序。实验表明,本文提出的线性误差模型对失真估计的误差率只有5\%,而采用该模型的码流截取相比于参考软件JSVM中的截取器有高达0.5dB的PSNR提升。
\item {基于PID控制思想的点播系统码率自适应算法} \\
自适应流媒体的另一个关键问题是传输过程中的码率调整策略,即在可用带宽不断变化的情况下,决定何时调整码率并确定调整到多少。本文基于经典的比例-积分-微分(PID)控制思想,提出了一个综合考虑带宽的历史状况、当前状态和未来趋势的码率自适应算法,既能充分利用带宽,传输较高的视频质量,又能减小带宽波动的影响,保证视频质量的平滑性。该算法在达尔文流媒体服务器中实现后与原有系统相比,平均质量等级提高了8.6\%,而质量波动降低了24.8\%,即不仅发送视频的质量更高,还取得了更好的平滑性。
\item {基于缓冲区分析的直播系统码率自适应算法} \\
在视频直播中,由于数据是实时产生的,其传输过程与视频点播有所不同。视频数据需要先上传到服务器,然后由服务器分发到观看者进行播放。本文为这个过程中的上传阶段增加了码率自适应的特性。首先通过详细分析系统整个传输过程中各个缓冲区的关系,建立了一个多缓冲区模型;然后把上述点播系统中用到的PID方法与多缓冲区模型相结合,提出了一个有效的码率自适应算法。相比于没有自适应的上传过程,带宽的利用率得到了提升,视频播放的连续性也得到了改善。
\end{enumerate}

本文的创新性算法和方案在实际的视频流媒体系统中得到了实现和一定的应用。

\section{未来工作展望}

以本文的成果作为基础,后续的研究可以扩展到其他类型的流媒体系统、立体视频和虚拟现实等更为复杂的应用场景,以及新一代视频编码标准。下面分三个方面对未来工作进行展望。
\begin{enumerate}
\item {更广泛的数据源优先级划分} \\
本文的码流截取工作只是针对可伸缩视频编码的码流,但事实上,流媒体系统中的源端数据多种多样。从目前的趋势来看,视频正朝着立体和全景等更加生动也更为复杂的方向发展,这其中除了传统的图像数据,还有深度数据、空间几何数据等,它们对视频质量的贡献方式与大小都差异很大。为了高效自适应地传输这些数据,弄清楚各个数据包对最终用户观感质量的重要性意义重大。这就是广义上的优先级划分问题。根据不同数据的性质考察其对信息重建的影响,为其建立失真模型并确定传输时的优先级,将是未来研究的内容之一。
\item {PID控制思想的扩展应用} \\
PID作为能有效控制和调节各种系统行为的基本思想,具有非常广阔的应用潜力。在本文中,PID用在了采用达尔文流媒体服务器的视频点播系统和用户生成内容的直播系统;在后续研究中,还可以将其应用到DASH等其他流媒体系统。此外,PID的关键在于反馈,即通过从系统获取信息来反作用于系统的输入,这些信息除了读取系统状态外,还可以从客户端或者观看者收集。尤其是对于虚拟现实应用来说,用户与系统更为紧密,可以成为这一研究方向的用武之地。进一步地,以本文对PID的应用作为启发,在后续工作中可以继续探索如何将非线性控制、最优控制等现代控制论中的原理和方法用于视频和多媒体相关领域。
\item {基于新一代视频编码标准的研究} \\
本文的研究主要是在上一代视频编码标准H.264/AVC及其可伸缩扩展SVC的基础上进行的。近两年间,最新一代的国际视频编码标准HEVC以及我国自主知识产权的新一代视频编码标准AVS2都已经发布,并且在不断的推广应用中。新的标准不仅进一步提高了压缩编码效率,而且仍然保持对可伸缩性的支持。下一步的研究可以基于新一代视频编码标准及其可伸缩扩展来进行,针对新标准的码流截取、编解码优化、以及在视频流媒体系统中的集成应用等方面均可以作为未来工作的一部分。
\end{enumerate}

目前移动直播和虚拟现实的热潮方兴未艾,视频流媒体作为其重要的技术支撑在很长一段时间内都将会有大量的需求和新的问题。服务器上的数据源处理、网络上的数据传输等本文所涉及的系统模块不会有大的改变,而且本文提出的思想方法很容易在新的应用场景或数据形式上适配。现有的这些工作和未来的研究将为人们突破时间和空间的限制更好感受这个世界带来更多可能。