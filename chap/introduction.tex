\chapter{绪论}

\section{研究背景和意义}

进入二十一世纪以来,人们表示和传递信息的媒介从文本、声音扩展到了图像、视频。如今视频已经成为人们生产和生活中不可缺少的部分。另一方面,在计算机技术和通信技术的推动下,人们对网络的依赖和要求也越来越高。
智能手机、平板电脑等移动设备的普及,Wi-Fi、4G等无线网络的覆盖,让人们能够随时随地访问互联网。这样的环境催生并促进了一项技术的蓬勃发展,这就是视频流媒体。

\subsection{视频流媒体的概念和应用}

视频流媒体简单来说就是通过网络在线播放视频。所谓流媒体,是指音视频等多媒体数据通过网络以连续稳定的流的形式传输到客户端的一系列技术、协议和方法的总称。在视频流媒体中,视频数据从服务器连续不断地传输到客户端,客户端可以一边接收一边播放,无需等待整个文件发送完毕。与下载方式相比,这种采用流式传输的视频播放具有显著的优点,包括:1)启动延迟大大缩短,用户可以在等待几秒或十几秒的缓冲后就立即开始观看;2)视频数据不在客户端长时间驻留,不仅节省了用户存储空间,也一定程度上避免了内容版权保护问题;3)支持数据的实时生成和获取,大大扩展了视频应用场景的范围。正是由于其优秀的特性,视频流媒体得到了广泛的应用,逐渐成为人们消费视频内容的主要形式。

视频流媒体应用按照其对实时性要求的不同可以分为点播和直播两大类。在点播应用中,内容提供商将预先制作好的视频放在服务器上,并发布内容的描述信息和链接,用户选择自己感兴趣的内容请求播放相应的视频。典型的例子是在线视频网站,如国外的YouTube、Netflix,国内的优酷、乐视等。大多在线教育网站如Coursera、网易公开课等也属于点播类应用。在直播应用中,视频数据则是将现场录制实时生成的,。远程监控、视频会议,以及现在互联网上流行的秀场、游戏等等。

\subsection{视频流媒体所面临的挑战}

在前所未有的机遇同时,视频流媒体也面临着诸多挑战。

一方面,从网络状况来看,目前既有几十kb/s速度的拨号上网,又有几十Mb/s速度的局域网或光纤接入方式,而且随着移动通信网和广播电视网与Internet的融合,无线与有线网络混联,使得带宽和信道质量更加不确定。在这样异构且波动的网络环境下,确定的视频流很难高效传输,也很难满足不同接入用户的需求。另一方面,从用户终端来看,运行视频应用的既可能是具有较强计算能力的PC机,也可以是只能进行低负荷运算的手机或PDA设备。显示装置既可能是高清的液晶显示器,也可能是只有QCIF大小(176×144)的手机屏幕。在这种情况下,传统编码技术得到的单一的视频流,很难满足不同用户的视频体验。例如,当视频码率过高的时候,手机用户根本无法正常播放;而较低的视频码率又会使得具有较强处理能力的PC用户不够满意。综上,只有视频流本身具有自适应性或可伸缩性,才能满足多样化的需求。

\section{本文研究内容和主要贡献}

本文结合视频流媒体所面临的挑战,针对其中的关键问题进行研究。首先,本文研究了基于可伸缩视频编码(SVC)的自适应流媒体技术,提出了新的可伸缩视频码流截取方案和传输过程中的码率自适应算法。其次,本文为最新的国际标准H.265/HEVC设计实现了一个高效解码器,并对其在x86和ARM通用处理器上的解码过程进行了研究,提出了新的优化算法,显著提高了解码速度。本文主要的工作和创新性贡献可以归纳为如下三个部分:

\begin{enumerate}
	\item {采用线性误差模型的码流截取方案}\\
	作为码率适应带宽波动的前提条件,视频流媒体中的数据源需要能够灵活调整。可伸缩视频编码通过将数据划分为基本层和增强层,通过丢弃增强层的数据包来实现即时码率变化。从完整的可伸缩码流中丢弃部分数据得到一个子流的过程称为码流截取。本文以最小化特定截取码率限制下的视频失真为目标,首先提出了一个线性误差模型来估计丢弃任意数据包组合带来的失真变化,然后利用它设计了一个贪心型算法来根据每个数据包的码率和失真影响对其赋优先级,作为截取过程中丢包的顺序。相比于参考软件,这一码流截取方案能够以更低的复杂度取得更高的视频质量。
	\item {基于PID控制思想的码率自适应算法}\\
	自适应流媒体的另一个关键问题是传输过程中的码率调整策略,即在可用带宽不断变化的情况下,决定何时调整码率并确定调整到多少。本文基于经典的比例-积分-微分(PID)控制思想,提出了一个综合考虑带宽的历史状况、当前状态和未来趋势的码率自适应算法,既能充分利用带宽,传输较高的视频质量,又能减小带宽波动的影响,保证视频质量的平滑性。该算法在点播和直播的实际测试中都表现出了很好的性能,而且很容易扩展到各种自适应流媒体系统。
	\item {针对新一代视频编码标准H.265/HEVC的解码优化}\\
	视频流媒体的最后一个阶段是码流在用户终端设备上的解码播放。为此,本文设计并实现了一个高度优化的H.265/HEVC解码器,将数据级和任务级并行方法相结合,显著提高了各个解码模块的计算效率以及整体解码速度。该解码器能够分别在主流PC和移动处理器上有效满足4K ($3840 \times 2160$) 和720p ($1280 \times 720$) 视频的实时解码需求,并在业界知名的视频流媒体平台迅雷看看上得到了验证,为新一代视频编码技术在流媒体中的应用和普及打下了坚实的基础。 
\end{enumerate}

\section{本文的结构安排}
本节说明本文的结构安排。