\chapter{绪论}
本章是绪论,也可以称为引言。

\section{研究背景和意义}
视频已经成为人们生产和生活不可缺少的一部分。而在计算机技术和网络技术的推动下,视频应用场景也变得非常复杂。尤其是在网络占主导的今天,在线或实时视频的随时随地观看、通过视频进行的随时随地通信,大大方便了人们的信息获取和交流沟通。然而,这种复杂的应用场景也为视频编码技术提出了新的挑战。见图1.1。

图1.1:视频应用的复杂场景
一方面,从网络状况来看,目前既有几十kb/s速度的拨号上网,又有几十Mb/s速度的局域网或光纤接入方式,而且随着移动通信网和广播电视网与Internet的融合,无线与有线网络混联,使得带宽和信道质量更加不确定。在这样异构且波动的网络环境下,确定的视频流很难高效传输,也很难满足不同接入用户的需求。另一方面,从用户终端来看,运行视频应用的既可能是具有较强计算能力的PC机,也可以是只能进行低负荷运算的手机或PDA设备。显示装置既可能是高清的液晶显示器,也可能是只有QCIF大小(176×144)的手机屏幕。在这种情况下,传统编码技术得到的单一的视频流,很难满足不同用户的视频体验。例如,当视频码率过高的时候,手机用户根本无法正常播放;而较低的视频码率又会使得具有较强处理能力的PC用户不够满意。在这样的情况下,只有视频流具有自适应性或可伸缩性,才能满足多样化的需求。
以前的一些视频编码标准,如MPEG-2、H.263、MPEG-4,在其高级特性中都提供了对可伸缩性的支持,但都将极大降低编码效率并增加解码复杂度。它们都没有被广泛应用。与之相反,一种称之为“联播”(Simulcast)的技术却在上述需求的推动下被许多视频网站(如优酷网)采用。这种技术原理很简单,就是预先编好不同码率的多个码流放在服务器上,根据网络和用户终端的情况选取其中合适的一个发送。这种方式提供了一定程度的适应性,但其缺点也是明显的。第一,编码、储存、传输多个码流使得时间和空间开销成倍增加;第二,这种方式一般只能提供很小的伸缩空间,如优酷网只提供了“普通”、“高清”、“超清”等几个选择。
2003年,国际电信联盟(ITU)和国际标准化组织(ISO)共同制定的新一代视频编码标准H.264/AVC(Advanced Video Coding)获得通过[1]。这一新标准不仅大大提高了编码效率,同时也强化了对网络的友好性,以顺应视频业务网络化的趋势。 H.264/AVC出台后,ITU和ISO共同成立的联合视频专家组(Joint Video Team)又开始着手新一代可伸缩视频编码(Scalable Video Coding,SVC)的标准化工作[2]。2007年,JVT的成果被作为H.264/AVC的扩展纳入标准化体系[3]。由于JVT制定的可伸缩编码框架依托于最新的H.264标准,其编码效率和实用性相比之前的可伸缩编码技术有了很大提高。下文提到SVC,一般特指H.264标准的SVC扩展。
SVC提出之后,大多数采用传统编码的视频业务,如数字电视、视频电话、视频会议、基于流媒体技术的在线视频观看等,都将从中得到改善。鉴于视频流媒体系统的普及,SVC进入这一应用领域的速度也将更快。流媒体是指采用流式传输的方式在网络上播放的媒体格式。与传统媒体播放不同,流媒体在播放前不需要下载整个文件内容,而是将数据分包,在传送的同时就可播放,只在开始时有缓冲延迟。流媒体系统的关键技术在于流式传输,即服务器能够将多媒体文件分割成包,按特定的顺序发给播放客户端,而客户端能够正确组合收到的数据流并解码出媒体数据。目前,流媒体技术已经得到了广泛的应用,除了互联网上常见的在线视频点播(VOD)、直播,流媒体也出现在远程教育和医疗、网络广告、实时视频会议、网络视频监控等诸多信息服务领域。
在视频流媒体领域,除了SVC之外,另外一个实现伸缩性和自适应性的技术DASH(Dynamic Adaptive Streaming over HTTP,即HTTP动态自适应流媒体)[4][5]也逐渐成为热点。DASH系统在服务器端提供不同码率的多个码流切片,通过在多码流之间切换来动态适应带宽波动。这类似于上文介绍的“联播“技术。虽然DASH相比SVC有成倍的开销和较小的伸缩性,但是由于采用HTTP传输,该技术的部署和分发相比SVC要容易很多。
在流媒体系统的研究中,质量控制和码流截取是两个重要的问题。质量控制是指调节所传输的视频的码率(随之也就改变了视频质量),使其能够符合当前带宽的变化,从而为用户提供尽可能好的观看体验。码流截取主要是指给定一个码率,如何从完整的SVC码流中截取出一个子流,使其在不超出给定码率限制的前提下具有尽可能好的质量。我们将在本文中对这两个问题进行讨论,详细给出其具体分析和相关工作。



\section{本文研究内容和主要贡献}
本节总结本文的研究内容和主要贡献。

\section{本文的结构安排}
本节说明本文的结构安排。