\chapter{绪论}

\section{研究背景和意义}

进入二十一世纪以来,人们表示和传递信息的媒介从文本、声音扩展到了图像、视频。如今视频已经成为人们生产和生活中不可缺少的部分。另一方面,在计算机技术和通信技术的推动下,人们对网络的依赖和要求也越来越高。
智能手机、平板电脑等移动设备的普及,Wi-Fi、4G等无线网络的覆盖,让人们能够随时随地访问互联网。这样的环境催生并促进了一项技术的蓬勃发展,这就是视频流媒体。

\subsection{视频流媒体的概念和应用}

视频流媒体简单来说就是通过网络在线播放视频。所谓流媒体,是指音视频等多媒体数据通过网络以连续稳定的流的形式传输到客户端的一系列技术、协议和方法的总称。在视频流媒体中,视频数据从服务器连续不断地传输到客户端,客户端可以一边接收一边播放,无需等待整个文件发送完毕。与下载方式相比,这种采用流式传输的视频播放具有显著的优点\supercite{Li2002},包括:1)启动延迟大大缩短,用户可以在等待几秒或十几秒的缓冲后就立即开始观看;2)视频数据不在客户端长时间驻留,不仅节省了用户存储空间,也一定程度上避免了内容版权保护问题;3)支持数据的实时生成和获取,大大扩展了视频应用场景的范围。正是由于其优秀的特性,视频流媒体得到了广泛的应用,逐渐成为人们消费视频内容的主要形式\supercite{Chen2013}。

视频流媒体应用按照其对实时性要求的不同可以分为点播和直播两大类。在点播应用中,内容提供商将预先制作好的视频放在服务器上,并发布内容的描述信息和链接,用户选择自己感兴趣的内容请求播放相应的视频。典型的例子是在线视频网站,如国外的YouTube、Netflix,国内的优酷、乐视等。大多在线教育网站如Coursera、网易公开课等也属于点播类应用。在直播应用中,视频数据则是通过现场录制实时生成的,上传到流媒体服务器之后再即时分发给观看者,具有很强的时效性。这类应用包括现在互联网上正兴起的生活、秀场、游戏类直播软件,以及已经很成熟的远程监控、视频会议等等。

网络条件的改善,采集与播放设备的普及,使得视频流媒体应用进入了一个高速增长的时期。根据Cisco的一项预测\footnote{http://www.cisco.com/c/en/us/solutions/collateral/service-provider/ip-ngn-ip-next-generation-network/white\_paper\_c11-481360.html},从2014年到2019年,视频流量在所有互联网流量中的占比将从64\%上升至80\%。路透社的一篇报道\footnote{http://www.reuters.com/article/us-internet-consumers-cisco-systems-idUSKBN0EL15E20140610}指出,在美国这一比例在2018年即将达到83\%。可见,视频流媒体将迎来一个高峰期。

\subsection{视频流媒体所面临的挑战}

在前所未有的机遇同时,视频流媒体也面临着诸多挑战。一方面,网络状况变得越来越复杂,传输可用带宽不可预知,如何保证视频数据及时高效地发送给每个用户成为了一个需要考虑的问题。另一方面,大部分用户终端设备的计算能力还很低,而视频分辨率在不断增大,且编解码的复杂度也越来越高,用户收到视频数据后的解码和显示所需计算量越来越大,快速实时解码变得越来越困难。下面对这两大挑战及其应对方法进行具体分析。

首先,从网络状况来看,目前既有几百kb/s速度的拨号或3G上网,又有几十Mb/s速度的局域网或光纤接入,而且随着移动通信网与Internet的融合,无线与有线网络混联,使得带宽和信道质量更加不确定。在这样异构且波动的网络环境下,确定的视频流很难高效传输,也很难满足不同接入用户的需求。举例来说,如果设置一个较低的码率,则对应的视频质量不高,达不到带宽充足用户的满意度;而如果设置较高的码率,则对低带宽的用户或者在带宽波动较大的情况下,会导致视频无法流畅播放,带来不好的用户体验。

其次,从用户终端来看,虽然前沿处理器计算能力不断在增强,但大部分用户的设备用的还是几年前的处理器,运算能力相当有限。尤其是对于智能手机和平板电脑而言,由于电池的制约,只能采用ARM架构的低功耗处理器。与此截然相反的是,现在的采集和显示设备像素成倍增加,视频分辨率早已突破了720p(1280x720)乃至1080p(1920x1080)的限制,向4K(3840x2160)发展。为了压缩这些高清和超高清的视频,视频编码的技术复杂度也越来越高。高分辨率的视频和高复杂度的编码技术,使得解码计算量大大增加。在通用处理器上的实时解码,成为了限制视频流媒体服务进一步改善的瓶颈。

考虑到视频流媒体的广泛应用,如何解决这些问题从而进一步改善流媒体服务质量,具有重要的现实意义。视频流媒体的研究既是工业界所关心的问题,也是学术界的一大热点。

\section{本文研究内容和主要贡献}

本文结合视频流媒体所面临的挑战,针对其中的关键问题进行研究。首先,为了应对异构网络环境中信道带宽的波动性,视频流媒体需要改变传统的单一码率传输,通过灵活调整发送的码率,来适应带宽变化。本文研究了基于可伸缩视频编码(SVC)的自适应流媒体技术,提出了新的可伸缩视频码流截取方案和传输过程中的码率自适应算法。其次,随着视频分辨率不断增大以及编解码复杂度的不断提高,在计算资源有限的终端设备上进行实时解码成为了流媒体应用中的一个重要问题。本文为最新的国际标准H.265/HEVC设计实现了一个高效解码器,并对其在x86和ARM通用处理器上的解码过程进行了研究,提出了新的优化算法,显著提高了解码速度。本文主要的工作和创新性贡献可以归纳为如下三个部分:

\begin{enumerate}
	\item {采用线性误差模型的码流截取方案}\\
	作为码率适应带宽波动的前提条件,视频流媒体中的数据源需要能够灵活调整。可伸缩视频编码通过将数据划分为基本层和增强层,通过丢弃增强层的数据包来实现即时码率变化。从完整的可伸缩码流中丢弃部分数据得到一个子流的过程称为码流截取。本文以最小化特定截取码率限制下的视频失真为目标,首先提出了一个线性误差模型来估计丢弃任意数据包组合带来的失真变化,然后利用它设计了一个贪心型算法来根据每个数据包的码率和失真影响对其赋优先级,作为截取过程中丢包的顺序。相比于参考软件,这一码流截取方案能够以更低的复杂度取得更高的视频质量。
	\item {基于PID控制思想的码率自适应算法}\\
	自适应流媒体的另一个关键问题是传输过程中的码率调整策略,即在可用带宽不断变化的情况下,决定何时调整码率并确定调整到多少。本文基于经典的比例-积分-微分(PID)控制思想,提出了一个综合考虑带宽的历史状况、当前状态和未来趋势的码率自适应算法,既能充分利用带宽,传输较高的视频质量,又能减小带宽波动的影响,保证视频质量的平滑性。该算法在点播和直播的实际测试中都表现出了很好的性能,而且很容易扩展到各种自适应流媒体系统。
	\item {针对新一代视频编码标准H.265/HEVC的解码优化}\\
	视频流媒体的最后一个阶段是码流在用户终端设备上的解码播放。为此,本文设计并实现了一个高度优化的H.265/HEVC解码器,将数据级和任务级并行方法相结合,显著提高了各个解码模块的计算效率以及整体解码速度。该解码器能够分别在主流PC和移动处理器上有效满足4K ($3840 \times 2160$) 和720p ($1280 \times 720$) 视频的实时解码需求,并在业界知名的视频流媒体平台迅雷看看上得到了验证,为新一代视频编码技术在流媒体中的应用和普及打下了坚实的基础。 
\end{enumerate}

\section{本文的结构安排}

本文共分为六章,后续章节具体内容安排如下。

第二章简要介绍视频流媒体领域的研究现状,并对可伸缩视频码流截取、码率自适应、解码器优化等方面的已有工作进行总结和分析。

第三章讨论采用线性误差模型的可伸缩视频码流截取方案。首先推导并验证线性误差模型,然后介绍采用该模型的失真估计方法和以码率失真影响为度量标准的优先级赋值算法,最后展现并分析所提出的码流截取方案的实验结果。

第四章讨论基于PID控制思想的码率自适应算法。首先对PID控制器做简单的介绍,然后将PID模型运用到视频传输中的码率自适应问题,提出了一个新颖的码率自适应算法。该算法被实现到实际流媒体系统中,其有效性通过对比实验得到了验证。

第五章讨论针对新一代视频编码标准的解码优化工作。首先给出了解码器的原型,然后介绍了采用数据级并行指令集扩展对特定解码模块进行加速的算法,最后介绍了多线程的设计和最终所取得的加速效果。

第六章总结全文内容并对未来工作和应用前景进行展望。