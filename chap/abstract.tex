\begin{cabstract}
近年来,计算机技术的发展使得移动设备逐渐普及,通信技术的发展使得互联网无处不在。在这样的条件下,通过网络随时随地观看视频成为可能,而且逐渐发展为视频内容消费的主要形式。这种无需先下载好全部视频数据而是一边下载一边播放的技术称为视频流媒体。视频流媒体的相关应用,如按需点播、实时直播、在线教育、视频会议等,已经成为人们生活中不可或缺的部分。在这前所未有的机遇同时,视频流媒体也面临着网络异构性和带宽波动的挑战。为应对这一挑战,视频流媒体系统需要能够根据不同的网络条件自动调整所发送的数据码率,以适应带宽的变化。本文\footnote{本研究得到国家自然科学基金(项目编号61271020)和国家科技支撑计划(项目编号2014BAK10B02) 资助。}从数据源和数据传输过程两方面入手,结合点播和直播这两种应用模式,对自适应视频流媒体中的关键技术进行了全面深入的研究。首先,本文针对可伸缩视频数据源提出了新的失真模型和码流截取方案,在支持可变码率的同时提供尽可能高的视频质量;其次,本文为视频点播系统设计了新的码率自动调整策略,用控制论的方法来解决适应带宽变化的问题;最后,本文详细分析了现在非常流行的视频直播系统的传输过程,结合直播的特点提出了数据上传时的码率自适应算法。本文主要的创新性贡献可以归纳为如下三个部分:
\begin{enumerate}
\item {采用线性误差模型的可伸缩视频码流截取方案}\\
作为码率适应带宽波动的前提条件,视频流媒体中的数据源需要能够灵活调整。可伸缩视频编码将数据划分为基本层和增强层,通过丢弃增强层的数据包来实现即时码率变化。从完整的可伸缩码流中丢弃部分数据得到一个子流的过程称为码流截取。本文以最小化特定截取码率限制下的视频失真为目标,首先提出了一个线性误差模型来估计丢弃任意数据包组合带来的失真变化,然后利用它设计了一个贪心型算法来根据每个数据包的码率和失真影响对其赋优先级,作为截取过程中丢包的顺序。相比于参考软件,这一码流截取方案能够在同样的复杂度和码率限制下取得更高的视频质量。
\item {基于PID控制思想的点播系统码率自适应算法}\\
自适应视频流媒体中的另一个关键问题是传输过程中的码率调整策略,即在可用带宽不断变化的情况下,决定何时调整码率并确定调整到多少。本文基于经典的比例-积分-微分(PID)控制思想,提出了一个综合考虑带宽的历史状况、当前状态和未来趋势的码率自适应算法,既能充分利用带宽,传输较高的视频质量,又能减小带宽波动的影响,保证视频质量的平滑性。该算法集成在了苹果公司QuickTime流媒体服务器的开源版本上,并在实际应用中取得了很好的效果。
\item {基于数据缓冲区分析的直播系统码率自适应算法}\\
在视频直播中,由于数据是实时产生的,其传输过程与视频点播有所不同。视频数据需要先上传到服务器,然后由服务器分发到观看者进行播放。本文为这个过程中的上传阶段增加了码率自适应的特性:首先通过详细分析系统整个传输过程中各个缓冲区的关系,建立了一个多缓冲区模型;然后把上述点播系统中用到的PID方法与多缓冲区模型相结合,提出0了一个有效的码率自适应算法。相比于没有自适应的上传过程,带宽的利用率得到了提升,观看者播放视频的连续性也得到了改善。
\end{enumerate}
\end{cabstract}

\begin{eabstract}
In recent years, the development of computer technology has made the mobile devices popular, and the communication technology has made the Internet accessible everywhere. Under such circumstances, watching videos at any time and any place becomes possible, and even an increasingly important way for people to consume video content. This is called video streaming, where the video can play as the data are being downloaded, before the entire file has been transmitted. Applications of video streaming, e.g., video on demand (VoD), live broadcasting, online education, and remote video conference, have become an indispensable part of people's life. Along with those opportunities, video streaming also has the big challenge brought about by the variety of networks. To cope with this challenge, the video streaming system should be able to adjust the video’s bitrate or quality according to the network condition. In other words, the video streaming system needs to be adaptive. In this paper, we investigate and solve the key problems in adaptive video streaming. First, focusing on the adaptive video streaming system based on the Scalable Video Coding (SVC) extension of the H.264/AVC video coding standard, this paper proposes a novel bitstream extraction scheme to provide highest possible video quality while supporting bitrate adjustment at the same time. Second, this paper designs a new rate adaptation algorithm for the VoD system, adjusting the bitrate to fit the current bandwidth from the control perspective. And finally, this paper analizes the transmission process of live video streaming in detail and proposes a rate adaptation algorithm for its uploading stage. The innovative contributions of this paper can be summarized as follows.
\begin{enumerate}
\item {Bitstream extraction scheme utilizing a linear error model}\\
As the prerequisite for adaptation, the data source in video streaming should be adjustable. SVC enables the flexible adjustment of bitrate by separating the video data into basic layer and enhancement layers and discarding the enhancement layer data packets as needed. The process of discarding some data from the whole SVC bitstream to obtain a substream is known as bitstream extraction. Aiming to minimize the distortion under the constraint of certain bitrate, a simple and effective linear error model is proposed and verified, which can be used to accurately estimate the distortion caused by discarding any combination of data packets from an SVC bitstream. Then utilizing this model, a greedy-like algorithm is designed to assign priority for each data packet according to its Rate-Distortion (R-D) impact, thus enabling optimized bitstream extraction. Comparing with the reference software, this extraction scheme achieves higher video quality with the same computational complexity and bitrate constraint.
\item {Rate adaptation algorithm for VoD systems based on the idea of PID control}\\
When and how to adjust the bitrate according to the bandwidth change is another important issue in adaptive video streaming. This paper proposes a rate adaptation algorithm based on the classical Proportional-Integral-Derivative (PID) controller. By monitoring and predicting past, current and future bandwidth information, the PID-based quality control algorithm is able to reduce quality fluctuation while still preserving a high quality level. The algorithm is integrated into the open source version of Apple's QuickTime streaming server and performs well in the practical applications.
\item {Rate adaptation algorithm for live streaming systems based on analysis of data buffers}\\
In live video streaming systems, the data are generated in real time; therefore, the transmission process is also different from that of VoD systems. The video data should first be uploaded to a server, then distributed to watchers by the server. This paper proposes to add adaptation to the uploading stage. A multi-buffer model is built based on analysis of the several data buffers during the transmission, and it is combined with the PID method to provide rate adaptation effectively. Compared with non-adaptive uploading, the bandwidth utilization is improved and the playback continuity is also increased.
\end{enumerate}
\end{eabstract}