\begin{cabstract}
视频流媒体系统已经逐渐向灵活码率和自适应质量的方向转变,以应对异构网络信道带宽的波动和播放终端的多样性。从目前的趋势来看,DASH(Dynamic Adaptive Streaming over HTTP,即HTTP动态自适应流媒体)和SVC(Scalable Video Coding,即可伸缩视频编码)将是视频流媒体中的主导技术。DASH系统在服务器端提供不同码率的多个码流切片,通过在多码流之间切换来动态适应带宽波动;SVC系统则只有一个高码率码流,通过丢弃其中一些数据来提供空间、时间和图像质量等方面的伸缩性。在上述流媒体系统的研究中,质量控制和码流截取是两个重要的问题。质量控制是指调节所传输的视频的码率(随之也就改变了视频质量),使其能够符合当前带宽的变化,从而为用户提供尽可能好的观看体验。码流截取主要是指给定一个码率,如何从完整的SVC码流中截取出一个子流,使其在不超出给定码率限制的前提下具有尽可能好的质量。本文将对着重对这两个问题进行论述。首先对视频编码和视频流媒体进行简介;然后介绍视频流媒体传输中的质量控制,包括控制目标、相关的控制算法和我们提出的PID的控制策略。接下来将介绍SVC码流截取的问题模型、已有的方法和我们基于线性误差模型的研究成果。最后,在总结已有工作的基础上我们提出了下一步研究计划。
\end{cabstract}

\begin{eabstract}
Video streaming system tend to be of flexible bitrate and adaptive quality, in order to meet the bandwidth fluctuation of heterogeneous network and the variety of client devices. DASH (Dynamic Adaptive Streaming over HTTP) and SVC (Scalable Video Coding) are seen to be the two leading technology. In DASH system, the server may transmit multiple streams with different bitrate, and adapts to the bandwidth by changing between these streams. On the contrary, SVC system can discard data in one coded video bit-stream, thus providing temporal, spatial and quality scalability. Included in the research of the video streaming are two important issues: quality control and bitstream extraction. The quality control of video streaming is to make the user’s watching experience as good as possible, by adjusting the transmission bitrate (therefor the video quality) according to the current bandwidth. The SVC bitstream extraction aims at selecting data packets from the whole stream, under the con¬straint of a given target bitrate, to form an “optimal” sub-stream, which represents video with best quality, or minimal distortion. This report will mainly focus on the above two issues. First, we give a brief introduction about video coding and video streaming. Then we talk about the quality control in video streaming, including the control target, the related control algorithms, and the PID-based strategy we proposed. Next, for SVC bitstream extraction, the problem model, existing solutions and our research results based on a linear error model are discussed. Finally, we summary our work and give the plan for further research.
\end{eabstract}