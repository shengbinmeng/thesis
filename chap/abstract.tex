\begin{cabstract}
近年来,计算机技术的发展使得移动设备逐渐普及,通信技术的发展使得互联网无处不在。在这样的条件下,通过网络随时随地观看视频成为可能,而且逐渐发展为视频内容消费的主要形式。这种无需先下载好全部视频数据而是一边下载一边播放的技术称为视频流媒体。视频流媒体的相关应用,如按需点播、实时直播、在线教育、视频会议等,已经成为人们生活中不可或缺的部分。在这前所未有的机遇同时,视频流媒体也面临着网络异构性和带宽波动的挑战。为应对这一挑战,视频流媒体系统需要能够根据不同的网络条件自动调整所发送的数据码率,以适应带宽的变化。本文\footnote{本研究得到国家自然科学基金(项目编号61271020)和国家科技支撑计划(项目编号2014BAK10B02) 资助。}从数据源和数据传输两方面入手,结合点播和直播两种应用模式,对自适应视频流媒体中的关键技术进行了较为全面深入的研究。首先,本文针对可伸缩视频数据源提出了新的失真模型和码流截取方案,在支持可变码率的同时提供尽可能高的视频质量;其次,本文为视频点播系统设计了新的码率自动调整策略,用控制论的方法来解决如何适应带宽变化的问题;最后,本文详细分析了现在非常流行的视频直播系统的传输过程,结合直播的特点提出了数据上传时的码率自适应算法。本文主要的创新性贡献可以归纳为如下三个部分:
\begin{enumerate}
\item {采用线性误差模型的可伸缩视频码流截取方案}\\
对于可伸缩视频数据源,本文首先推导并验证了一个线性误差模型,用于准确估计丢弃任意数据包组合带来的失真变化;然后采用该模型设计了一个贪心型算法来根据每个数据包的码率和失真影响对其赋优先级,作为截取子流时丢弃数据包的顺序。相比于参考软件,这一码流截取方案能够在同样的复杂度和码率限制下取得更高的视频质量。
\item {基于PID控制思想的点播系统码率自适应算法}\\
本文基于经典的比例-积分-微分(PID)控制思想,为视频点播系统的数据传输过程提出了一个综合考虑带宽的历史状况、当前状态和未来趋势的码率自适应算法,既能充分利用带宽,传输较高的视频质量,又能减小带宽波动的影响,保证视频质量的平滑性。该算法集成在了苹果公司QuickTime流媒体服务器的开源版本上,将发送的视频平均质量提高了8.6\%,质量波动降低了24.8\%。
\item {基于缓冲区分析的直播系统码率自适应算法}\\
为给视频直播中的数据上传阶段增加码率自适应的特性,本文首先详细分析了系统整个传输过程中各个缓冲区的关系,建立了一个多缓冲区模型;然后把上述点播系统中用到的PID方法与多缓冲区模型相结合,提出了一个有效的上传过程码率自适应算法。相比于没有自适应的上传过程,带宽的利用率得到了提升,视频播放的连续性也得到了改善。
\end{enumerate}
\end{cabstract}

\begin{eabstract}
In recent years, the development of computer technology has made the mobile devices popular, and the communication technology has made the Internet accessible everywhere. Under such circumstances, watching videos at any time and any place becomes possible, and even an increasingly important way for people to consume video content. This is called video streaming, where the video can play as the data are being transmitted, before the entire file has been downloaded. Applications of video streaming, e.g., video on demand (VoD), live broadcasting, online education, and remote video conference, have become an indispensable part of people's life. Along with those opportunities, video streaming also has the big challenge brought about by the variety of networks. To cope with this challenge, the video streaming system should be able to adjust the video's bitrate or quality according to the network condition. In other words, the video streaming system needs to be adaptive. In this paper, we investigate and solve the key problems in adaptive video streaming. First, focusing on the adaptive video streaming system based on the Scalable Video Coding (SVC) extension of the H.264/AVC video coding standard, this paper proposes a novel bitstream extraction scheme to provide highest possible video quality while supporting bitrate adjustment at the same time. Second, this paper designs a new rate adaptation algorithm for the VoD system, adjusting the bitrate to fit the current bandwidth from the control perspective. And finally, this paper analizes the transmission process of live video streaming in detail and proposes a rate adaptation algorithm for its uploading stage. The innovative contributions of this paper can be summarized as follows.
\begin{enumerate}
\item {Bitstream extraction scheme utilizing a linear error model}\\
For scalable video data source, a simple and effective linear error model is proposed and verified, which can be used to accurately estimate the distortion caused by discarding any combination of data packets from an SVC bitstream. Then utilizing this model, a greedy-like algorithm is designed to assign priority for each data packet according to its Rate-Distortion (R-D) impact, thus enabling optimized bitstream extraction. Comparing with the reference software, this extraction scheme achieves higher video quality with the same computational complexity and bitrate constraint.
\item {Rate adaptation algorithm for VoD systems based on the idea of PID control}\\
Based on the classical Proportional-Integral-Derivative (PID) controller, a rate adaptation algorithm is proposed for the data transmission process of VoD systems. By monitoring and predicting past, current and future bandwidth information, the algorithm is able to reduce quality fluctuation while still preserving a high quality level. Integrated into the open source version of Apple's QuickTime streaming server, this algorithm increases the streamed video's quality by 8.6\% and decreases the quality variance by 24.8\% at the same time.
\item {Rate adaptation algorithm for live streaming systems based on analysis of data buffers}\\
This paper also proposes to add adaptation for the data uploading stage of live video streaming systems. A multi-buffer model is built based on analysis of the several data buffers during the transmission, and it is combined with the PID method to provide rate adaptation effectively. Compared with non-adaptive uploading, the bandwidth utilization is increased and the playback continuity is improved.
\end{enumerate}
\end{eabstract}