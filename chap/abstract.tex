\begin{cabstract}
近年来,计算机技术的发展使得移动设备逐渐普及,通信技术的发展又使得互联网无处不在。在这样的条件下,通过网络随时随地观看视频成为可能,而且逐渐发展为视频内容消费的主要形式。这种无需先下载好全部视频数据而是一边下载一边播放的技术称为视频流媒体。视频流媒体的相关应用,如按需点播(VoD)、实时直播、在线教育、远程会议等,已经成为人们生活中不可或缺的部分。在这前所未有的机遇同时,视频流媒体也面临着网络环境复杂多样、终端设备计算能力有限等许多挑战。本文\footnote{本研究得到国家自然科学基金(编号60902004、61271020) 和国家基础研究项目(973 计划,编号2009CB320907) 资助。}结合视频流媒体所面临的挑战,针对其中的关键问题进行研究。首先,为了应对异构网络环境中信道带宽的波动性,视频流媒体需要改变传统的单一码率传输,通过灵活调整发送的码率,来适应带宽变化。本文研究了基于可伸缩视频编码(SVC)的自适应流媒体技术,提出了新的可伸缩视频码流截取方案和传输过程中的码率自适应算法。其次,随着视频分辨率不断增大以及编解码复杂度的不断提高,在计算资源有限的终端设备上进行实时解码成为了流媒体应用中的一个重要问题。本文为最新的国际标准H.265/HEVC设计实现了一个高效解码器,并对其在x86和ARM通用处理器上的解码过程进行了研究,提出了新的优化算法,显著提高了解码速度。本文主要的工作和创新性贡献可以归纳为如下三个部分:
\begin{enumerate}
\item {采用线性误差模型的码流截取方案}\\
作为码率适应带宽波动的前提条件,视频流媒体中的数据源需要能够灵活调整。可伸缩视频编码通过将数据划分为基本层和增强层,通过丢弃增强层的数据包来实现即时码率变化。从完整的可伸缩码流中丢弃部分数据得到一个子流的过程称为码流截取。本文以最小化特定截取码率限制下的视频失真为目标,首先提出了一个线性误差模型来估计丢弃任意数据包组合带来的失真变化,然后利用它设计了一个贪心型算法来根据每个数据包的码率和失真影响对其赋优先级,作为截取过程中丢包的顺序。相比于参考软件,这一码流截取方案能够以更低的复杂度取得更高的视频质量。
\item {基于PID控制思想的码率自适应算法}\\
自适应流媒体的另一个关键问题是传输过程中的码率调整策略,即在可用带宽不断变化的情况下,决定何时调整码率并确定调整到多少。本文基于经典的比例-积分-微分(PID)控制思想,提出了一个综合考虑带宽的历史状况、当前状态和未来趋势的码率自适应算法,既能充分利用带宽,传输较高的视频质量,又能减小带宽波动的影响,保证视频质量的平滑性。该算法在点播和直播的实际测试中都表现出了很好的性能,而且很容易扩展到各种自适应流媒体系统。
\item {针对新一代视频编码标准H.265/HEVC的解码优化}\\
视频流媒体的最后一个阶段是码流在用户终端设备上的解码播放。为此,本文设计并实现了一个高度优化的H.265/HEVC解码器,将数据级和任务级并行方法相结合,显著提高了各个解码模块的计算效率以及整体解码速度。该解码器能够分别在主流PC和移动处理器上有效满足4K ($3840 \times 2160$) 和720p ($1280 \times 720$) 视频的实时解码需求,并在业界知名的视频流媒体平台迅雷看看上得到了验证,为新一代视频编码技术在流媒体中的应用和普及打下了坚实的基础。 
\end{enumerate}
\end{cabstract}

\begin{eabstract}
In recent years, the development of computer technology has made the mobile devices popular, and the communication technology has made the Internet accessible everywhere. Under such circumstances, watching videos at any time and any place becomes possible, and even an increasingly important way for people to consume video content. This is called video streaming, where the video can play as the data are being downloaded, before the entire file has been transmitted. Applications of video streaming, e.g., video on demand (VoD), live broadcasting, online education, and remote video conference, have become an indispensable part of people's life. Together with those opportunities, video streaming also has some challenges, such as the variety of networks, and the limited computing capacity of client devices. In this paper, we investigate and solve the key problems in video streaming, to cope with these challenges. First, to cope with the channel bandwidth fluctuation in heterogeneous network environment, a video streaming system should be able to adjust sending bitrate according to the bandwidth change, instead of transmitting at a fixed bitrate. Focusing on the adaptive streaming technology based on Scalable Video Coding (SVC), the scalable extension of the H.264/AVC video coding standard, this paper proposes a novel bitstream extraction scheme and a rate adaptation algorithm used in transmission. Second, with the increase of the video's resolution and coding complexity, it becomes a nontrivial task to achieve real-time decoding in a device with limited computing resources. This paper designs and implements an efficient decoder for the latest international coding standard H.265/HEVC. By analyzing the decoding process in general processers of x86 and ARM architecture, new optimization algorithms are proposed and the decoding speed is significantly increased. The works and innovative contributions of this paper can be summarized as follows.
\begin{enumerate}
\item {Bitstream extraction scheme utilizing a linear error model}\\
As the prerequisite for the rate adaptation, the data source in video streaming should be adjustable. SVC enables the flexible adjust of bitrate by separating the video data into basic layer and enhancement layers and discarding the enhancement layer data packets as needed. The process of discarding some data from the whole SVC bitstream to obtain a substream is known as bitstream extraction. Aiming to minimize the distortion under the constraint of certain bitrate, a simple and effective linear error model is proposed and verified, which can be used to accurately estimate the distortion caused by discarding any combination of data packets from an SVC bitstream. Then utilizing this model, a greedy-like algorithm is designed to assign priority for each data packet according to its Rate-Distortion (R-D) impact, thus enabling optimized bitstream extraction. Comparing with the reference software, this extraction scheme achieves higher video quality with lower computational complexity.
\item {Rate adaptation algorithm based on the idea of PID control}\\
When and how to adjust the bitrate according to the bandwidth change is another important issue in adaptive video streaming. This paper proposes a rate adaptation algorithm based on the classical Proportional-Integral-Derivative (PID) controller. By monitoring and predicting past, current and future bandwidth information, the PID-based quality control algorithm is able to reduce quality fluctuation while still preserving a high quality level. The algorithm performs well in the test of both VoD and live broadcasting scenario, and should be universally effective in all kinds of adaptive streaming systems.
\item {Decoder optimization for the new-generation coding standard H.265/HEVC}\\
The final stage of video streaming is decoding and playing on the user's devices. For that, this paper designes and implements a highly optimized H.265/HEVC decoder, which combines data-level and task-level parallelism and improves each decoding module's computing efficiency as well as the overall decoding speed. The decoder can meet the requirement of real-time decoding for 4K ($3840 \times 2160$) and 720p ($1280 \times 720$) videos on mainstream PC and mobile CPUs, respectively. Already used by the famous video streaming service provider Xunlei KanKan, the optimized decoder is well prepared to promote the application of the latest coding standard in video streaming.
\end{enumerate}
\end{eabstract}