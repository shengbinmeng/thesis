\chapter{致谢}

这篇论文的完成,也意味着我五年的博士研究生之路即将走到终点。在这期间,我经历了很多,也收获了很多。
我的工作成果和个人成长离不开许多人的帮助和支持,在此我想向他们表达谢意。

首先衷心感谢我的导师郭宗明老师。感谢郭老师这几年来对我的培养、关怀、支持和影响。郭老师在我学术研究方面高屋建瓴的指导,在我为人处事方面言传身教的影响,都让我受益匪浅。
在我投稿被拒时帮我分析原因,鼓励我;每学期找我谈话,总结经验。在我取得成果时发来祝贺、让我出国开会;在我闯祸犯错时给予我批评或是谅解……这一切我都铭记在心。
郭老师随和豁达待人、认真严谨做事的风格,对研究室团队建设和体育锻炼的重视,都深深地影响了我。

其次要特别感谢孙俊老师。孙老师让我接受工程的训练,提高了我的动手能力;领我进入学术的大门,开启了我的科研道路。
从选取研究题目,到探索新的方法,再到实验验证和论文写作,我博士研究中的每一步都得益于和孙老师的交流讨论。我所取得的每一项科研成果,都是在孙老师的指导和帮助下完成。
一次次指导我修改论文,一次次指导我完善文档。精益求精的态度让我深受启发。

感谢数字视频研究室的刘家瑛、张行功、李晓龙等各位老师,我从他们那里得到过不少宝贵的意见和过来人的经验。感谢北京大学计算机科学技术研究所的彭宇新、肖建国、汤帜、赵东岩、陈晓鸥等各位老师,清华大学自动化系的季向阳老师,北京大学信息科学技术学院的马思伟、熊瑞勤两位老师,感谢他们在我博士不同阶段的指导和帮助。感谢计算机所综合办公室的戴永宁、韩玉晶、姚春夏、张猛等各位老师,感谢这些老师为我学业和科研的顺利进行提供保障。

感谢陈科吉、舒爽、范英明、张奇、王杰西、林镇安、李马丁、耿玉峰等数字视频研究室的同学们,以及曾在研究室待过如今已经离开这里的邓戬峰、段一舟、颜乐驹、刘森、周燕萍、任杰、周超、汤凯、白蔚、王一磊、张文尧等人,还有与我同班或同宿舍的刘丙双、姚金戈等人。感谢他们给我带来工作上的协助和生活中的友谊。

感谢我的家人,尤其是我的父母和妻子。感谢他们一直以来的牵挂和支持,他们的爱是我力量的源泉。