\chapter{致谢}

这篇论文的完成,也意味着我五年的博士研究生之路即将走到终点。在这期间,我经历了很多,也收获了很多。
我的工作成果和个人成长离不开许多人的帮助和支持,在此我想向他们表达谢意。

首先衷心感谢我的导师郭宗明老师。感谢郭老师这几年来对我无私的培养和关怀。郭老师在我学术研究方面高屋建瓴的指导,在我为人处事方面微言大义的教诲,都让我受益匪浅。
郭老师生活中随和豁达、工作上认真严谨的风格,对科研和教育事业的热情,都深深地影响了我。郭老师百忙之中仍重视和支持研究室的团体活动和体育锻炼,让我们得以健康成长。
郭老师不仅为我传道、授业、解惑,还给予我信任、期望和鼓励,在人生重要时期遇到这样一位导师真是我的幸运。

其次要特别感谢我的指导老师孙俊老师。感谢孙老师让我接受工程的训练,提高了我的动手能力;领我进入学术的大门,开启了我的科研道路。
从选取研究题目,到探索新的方法,再到实验验证和论文写作,我的研究工作无不得益于和孙老师的交流讨论。我所取得的科研成果也都是在孙老师的指导和帮助下完成。
孙老师精益求精的态度让我深受感染,严格细致的要求督促我不断进步,科研上的悉心指导,生活中的言传身教,这些对我而言都是一笔巨大的财富。

感谢数字视频研究室的刘家瑛、张行功、李晓龙等各位老师,我从他们那里得到过不少有用的建议和过来人的经验。感谢北京大学计算机科学技术研究所的彭宇新、肖建国、汤帜、赵东岩、陈晓鸥等各位老师,清华大学自动化系的季向阳老师,北京大学信息科学技术学院的马思伟、熊瑞勤两位老师,感谢他们在我博士不同阶段的指导和帮助。感谢计算机所综合办公室的戴永宁、韩玉晶、张猛、姚春霞等各位老师,感谢这些老师为我学业和科研的顺利进行提供保障。

感谢陈科吉、舒爽、范英明、张奇、王杰西、林镇安、钟明、李马丁、耿玉峰等数字视频研究室的同学们,以及曾在研究室待过如今已经离开这里的邓戬峰、段一舟、颜乐驹、刘森、周燕萍、任杰、周超、汤凯、白蔚、王一磊、张文尧等人,还有与我同班或同宿舍的刘丙双、姚金戈等人。感谢他们给我带来工作上的协助和生活中的友谊。

感谢我的家人,尤其是我的父母和妻子。感谢他们一直以来的牵挂和支持,他们的爱是我力量的源泉。

最后,非常感谢各位评审老师拨冗评阅此论文并提供宝贵意见。